
 \textbf  {[1] Deep Hashing Based on VAE-GAN for Efficient Similarity Retrieval}

In this research, a VAE-GAN based hashing architecture for fast photo retrieval was investigated. The method combines a Variational autoencoder (VAE) with a Generative adversarial network to generate content-preserving pictures for pairwise hashing learning (GAN). Accepting actual and synthesised images in paired form, an adversarial generative approach is utilised to learn a semantic perserving feature mapping model.
Each pairwise image feature vector is converted to hash codes, which are subsequently used in a pairwise ranking loss to maintain relative image similarity. TensorFlow, one of the most widely used deep learning frameworks, should be utilised to implement VGH. All of the weights are set up by xavier. In the midst of putting together an image. On Adam, the VAE-GAN models are trained on all datasets.

 \textbf { [2] Recurrent Topic Transition GAN for Visual Paragraph Generation}
 
The RTT-GAN (Recurrent Topic Transition Generative Adversarial Network) that has been proposed builds an adversarial framework between a structured paragraph generator and multi-level paragraph discriminators.
To create sentences frequently, the paragraph generator uses region-based visual and language attention strategies at each level. Multi-level adversarial discriminators assess the quality of generated paragraph.sentences on two levels: sentence plausibility and paragraph topic-transition coherence.


 \textbf {[3] Natural Language Processing approaches for Text2Image using Machine Learning Algorithms}

A word cloud was used in this research report to help the reader understand the artist's concept. MLA - Random Forest was also used, with a classification accuracy of 78 on the dataset in question. Transform the data source Creating an image from text data The morphological stage is the first and is concerned with the various forms of words. Syntactical analysis will focus on the various relationships between the words in the sentence in the next stage. We'll move on to semantics when we've mastered some synthetic structures. As a result, semantics is concerned with figuring out what something means. The highest level of abstraction is pragmatics, which is the ultimate stage.




 \textbf {[4] Image-to-Image Translation with Conditional Adversarial Networks}

\par Demonstrate that, among other things, the GAN approach can be used to synthesise photos from label maps, reconstruct objects from edge maps, and colourize images. According to the findings of this research, conditional adversarial networks appear to be a promising answer for many image-to-image translation difficulties, particularly those involving highly organised graphical outputs. These networks learn a loss that is specific to the task and data, making them effective in a variety of scenarios.

