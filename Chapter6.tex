\section{Data analysis}


discriminator_model.py 
import torch
import torch.nn as nn


class CNNBlock(nn.Module):
    def __init__(self, in_channels, out_channels, stride):
        super(CNNBlock, self).__init__()
        self.conv = nn.Sequential(
            nn.Conv2d(
                in_channels, out_channels, 4, stride, 1, bias=False, padding_mode="reflect"
            ),
            nn.BatchNorm2d(out_channels),
            nn.LeakyReLU(0.2),
        )

    def forward(self, x):
        return self.conv(x)


class Discriminator(nn.Module):
    def __init__(self, in_channels=3, features=[64, 128, 256, 512]):
        super().__init__()
        self.initial = nn.Sequential(
            nn.Conv2d(
                in_channels * 2,
                features[0],
                kernel_size=4,
                stride=2,
                padding=1,
                padding_mode="reflect",
            ),
            nn.LeakyReLU(0.2),
        )

        layers = []
        in_channels = features[0]
        for feature in features[1:]:
            layers.append(
                CNNBlock(in_channels, feature, stride=1 if feature == features[-1] else 2),
            )
            in_channels = feature

        layers.append(
            nn.Conv2d(
                in_channels, 1, kernel_size=4, stride=1, padding=1, padding_mode="reflect"
            ),
        )

        self.model = nn.Sequential(*layers)

    def forward(self, x, y):
        x = torch.cat([x, y], dim=1)
        x = self.initial(x)
        x = self.model(x)
        return x


def test():
    x = torch.randn((1, 3, 256, 256))
    y = torch.randn((1, 3, 256, 256))
    model = Discriminator(in_channels=3)
    preds = model(x, y)
    print(model)
    print(preds.shape)


if __name__ == "__main__":
    test()

requirements.txt
albumentations==1.1.0
certifi==2020.6.20
colorama==0.4.4
cycler==0.11.0
decorator==4.4.2
imageio==2.13.1
joblib==1.1.0
kiwisolver==1.3.1
matplotlib==3.3.4
networkx==2.5.1
numpy==1.19.5
opencv-python-headless==4.5.4.60
Pillow==8.4.0
pyparsing==3.0.6
python-dateutil==2.8.2
PyWavelets==1.1.1
PyYAML==6.0
qudida==0.0.4
scikit-image==0.17.2
scikit-learn==0.24.2
scipy==1.5.4
six==1.16.0
threadpoolctl==3.0.0
tifffile==2020.9.3
torch==1.10.0
torchvision==0.11.1
tqdm==4.62.3
typing_extensions==4.0.1
wincertstore==0.2


generator_model.py 

import torch
import torch.nn as nn


class Block(nn.Module):
    def __init__(self, in_channels, out_channels, down=True, act="relu", use_dropout=False):
        super(Block, self).__init__()
        self.conv = nn.Sequential(
            nn.Conv2d(in_channels, out_channels, 4, 2, 1, bias=False, padding_mode="reflect")
            if down
            else nn.ConvTranspose2d(in_channels, out_channels, 4, 2, 1, bias=False),
            nn.BatchNorm2d(out_channels),
            nn.ReLU() if act == "relu" else nn.LeakyReLU(0.2),
        )

        self.use_dropout = use_dropout
        self.dropout = nn.Dropout(0.5)
        self.down = down

    def forward(self, x):
        x = self.conv(x)
        return self.dropout(x) if self.use_dropout else x


class Generator(nn.Module):
    def __init__(self, in_channels=3, features=64):
        super().__init__()
        self.initial_down = nn.Sequential(
            nn.Conv2d(in_channels, features, 4, 2, 1, padding_mode="reflect"),
            nn.LeakyReLU(0.2),
        )
        self.down1 = Block(features, features * 2, down=True, act="leaky", use_dropout=False)
        self.down2 = Block(
            features * 2, features * 4, down=True, act="leaky", use_dropout=False
        )
        self.down3 = Block(
            features * 4, features * 8, down=True, act="leaky", use_dropout=False
        )
        self.down4 = Block(
            features * 8, features * 8, down=True, act="leaky", use_dropout=False
        )
        self.down5 = Block(
            features * 8, features * 8, down=True, act="leaky", use_dropout=False
        )
        self.down6 = Block(
            features * 8, features * 8, down=True, act="leaky", use_dropout=False
        )
        self.bottleneck = nn.Sequential(
            nn.Conv2d(features * 8, features * 8, 4, 2, 1), nn.ReLU()
        )

        self.up1 = Block(features * 8, features * 8, down=False, act="relu", use_dropout=True)
        self.up2 = Block(
            features * 8 * 2, features * 8, down=False, act="relu", use_dropout=True
        )
        self.up3 = Block(
            features * 8 * 2, features * 8, down=False, act="relu", use_dropout=True
        )
        self.up4 = Block(
            features * 8 * 2, features * 8, down=False, act="relu", use_dropout=False
        )
        self.up5 = Block(
            features * 8 * 2, features * 4, down=False, act="relu", use_dropout=False
        )
        self.up6 = Block(
            features * 4 * 2, features * 2, down=False, act="relu", use_dropout=False
        )
        self.up7 = Block(features * 2 * 2, features, down=False, act="relu", use_dropout=False)
        self.final_up = nn.Sequential(
            nn.ConvTranspose2d(features * 2, in_channels, kernel_size=4, stride=2, padding=1),
            nn.Tanh(),
        )

    def forward(self, x):
        d1 = self.initial_down(x)
        d2 = self.down1(d1)
        d3 = self.down2(d2)
        d4 = self.down3(d3)
        d5 = self.down4(d4)
        d6 = self.down5(d5)
        d7 = self.down6(d6)
        bottleneck = self.bottleneck(d7)
        up1 = self.up1(bottleneck)
        up2 = self.up2(torch.cat([up1, d7], 1))
        up3 = self.up3(torch.cat([up2, d6], 1))
        up4 = self.up4(torch.cat([up3, d5], 1))
        up5 = self.up5(torch.cat([up4, d4], 1))
        up6 = self.up6(torch.cat([up5, d3], 1))
        up7 = self.up7(torch.cat([up6, d2], 1))
        return self.final_up(torch.cat([up7, d1], 1))


def test():
    x = torch.randn((1, 3, 256, 256))
    model = Generator(in_channels=3, features=64)
    preds = model(x)
    print(preds.shape)


if __name__ == "__main__":
    test()




Train.py
import torch
from utils import save_checkpoint, load_checkpoint, save_some_examples
import torch.nn as nn
import torch.optim as optim
import os
os.environ["KMP_DUPLICATE_LIB_OK"]="TRUE"

import config
from dataset import MapDataset
from generator_model import Generator
from discriminator_model import Discriminator
from torch.utils.data import DataLoader
from tqdm import tqdm
from torchvision.utils import save_image

torch.backends.cudnn.benchmark = True


def train_fn(
    disc, gen, loader, opt_disc, opt_gen, l1_loss, bce, g_scaler, d_scaler,
):
    loop = tqdm(loader, leave=True)

    for idx, (x, y) in enumerate(loop):
        x = x.to(config.DEVICE)
        y = y.to(config.DEVICE)

        # Train Discriminator
        with torch.cuda.amp.autocast():
            y_fake = gen(x)
            D_real = disc(x, y)
            D_real_loss = bce(D_real, torch.ones_like(D_real))
            D_fake = disc(x, y_fake.detach())
            D_fake_loss = bce(D_fake, torch.zeros_like(D_fake))
            D_loss = (D_real_loss + D_fake_loss) / 2

        disc.zero_grad()
        d_scaler.scale(D_loss).backward()
        d_scaler.step(opt_disc)
        d_scaler.update()

        # Train generator
        with torch.cuda.amp.autocast():
            D_fake = disc(x, y_fake)
            G_fake_loss = bce(D_fake, torch.ones_like(D_fake))
            L1 = l1_loss(y_fake, y) * config.L1_LAMBDA
            G_loss = G_fake_loss + L1

        opt_gen.zero_grad()
        g_scaler.scale(G_loss).backward()
        g_scaler.step(opt_gen)
        g_scaler.update()

        if idx % 10 == 0:
            loop.set_postfix(
                D_real=torch.sigmoid(D_real).mean().item(),
                D_fake=torch.sigmoid(D_fake).mean().item(),
            )


def main():
    disc = Discriminator(in_channels=3).to(config.DEVICE)
    gen = Generator(in_channels=3, features=64).to(config.DEVICE)
    opt_disc = optim.Adam(disc.parameters(), lr=config.LEARNING_RATE, betas=(0.5, 0.999),)
    opt_gen = optim.Adam(gen.parameters(), lr=config.LEARNING_RATE, betas=(0.5, 0.999))
    BCE = nn.BCEWithLogitsLoss()
    L1_LOSS = nn.L1Loss()

    if config.LOAD_MODEL:
        load_checkpoint(
            config.CHECKPOINT_GEN, gen, opt_gen, config.LEARNING_RATE,
        )
        load_checkpoint(
            config.CHECKPOINT_DISC, disc, opt_disc, config.LEARNING_RATE,
        )

    train_dataset = MapDataset(root_dir=config.TRAIN_DIR)
    train_loader = DataLoader(
        train_dataset,
        batch_size=config.BATCH_SIZE,
        shuffle=True,
        num_workers=config.NUM_WORKERS,
    )
    g_scaler = torch.cuda.amp.GradScaler()
    d_scaler = torch.cuda.amp.GradScaler()
    val_dataset = MapDataset(root_dir=config.VAL_DIR)
    val_loader = DataLoader(val_dataset, batch_size=1, shuffle=False)

    for epoch in range(config.NUM_EPOCHS):
        train_fn(
            disc, gen, train_loader, opt_disc, opt_gen, L1_LOSS, BCE, g_scaler, d_scaler,
        )

        if config.SAVE_MODEL and epoch % 5 == 0:
            save_checkpoint(gen, opt_gen, filename=config.CHECKPOINT_GEN)
            save_checkpoint(disc, opt_disc, filename=config.CHECKPOINT_DISC)

        save_some_examples(gen, val_loader, epoch, folder="evaluation")


if __name__ == "__main__":
    main()


Utils.py
import torch
import config
from torchvision.utils import save_image

def save_some_examples(gen, val_loader, epoch, folder):
    x, y = next(iter(val_loader))
    x, y = x.to(config.DEVICE), y.to(config.DEVICE)
    gen.eval()
    with torch.no_grad():
        y_fake = gen(x)
        y_fake = y_fake * 0.5 + 0.5  # remove normalization#
        save_image(y_fake, folder + f"/y_gen_{epoch}.png")
        save_image(x * 0.5 + 0.5, folder + f"/input_{epoch}.png")
        if epoch == 1:
            save_image(y * 0.5 + 0.5, folder + f"/label_{epoch}.png")
    gen.train()


def save_checkpoint(model, optimizer, filename="my_checkpoint.pth.tar"):
    print("=> Saving checkpoint")
    checkpoint = {
        "state_dict": model.state_dict(),
        "optimizer": optimizer.state_dict(),
    }
    torch.save(checkpoint, filename)


def load_checkpoint(checkpoint_file, model, optimizer, lr):
    print("=> Loading checkpoint")
    checkpoint = torch.load(checkpoint_file, map_location=config.DEVICE)
    model.load_state_dict(checkpoint["state_dict"])
    optimizer.load_state_dict(checkpoint["optimizer"])

    # If we don't do this then it will just have learning rate of old checkpoint
    # and it will lead to many hours of debugging \:
    for param_group in optimizer.param_groups:
        param_group["lr"] = lr
        




\subsection{Data Preprocessing}



